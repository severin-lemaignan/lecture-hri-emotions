%!TEX program = xelatex

\documentclass[compress]{beamer}
%--------------------------------------------------------------------------
% Common packages
%--------------------------------------------------------------------------

\definecolor{links}{HTML}{663000}
\hypersetup{colorlinks,linkcolor=,urlcolor=links}

\usepackage[english]{babel}
\usepackage{pgfpages} % required for notes on second screen
\usepackage{graphicx}

\usepackage{pdfpcnotes}

\usepackage{multicol}

\usepackage{tabularx,ragged2e}
\usepackage{booktabs}

\usetheme{hri}

% Display the navigation bullet even without subsections
\usepackage{remreset}% tiny package containing just the \@removefromreset command
\makeatletter
\@removefromreset{subsection}{section}
\makeatother
\setcounter{subsection}{1}

\makeatletter
\let\beamer@writeslidentry@miniframeson=\beamer@writeslidentry
\def\beamer@writeslidentry@miniframesoff{%
  \expandafter\beamer@ifempty\expandafter{\beamer@framestartpage}{}% does not happen normally
  {%else
    % removed \addtocontents commands
    \clearpage\beamer@notesactions%
  }
}
\newcommand*{\miniframeson}{\let\beamer@writeslidentry=\beamer@writeslidentry@miniframeson}
\newcommand*{\miniframesoff}{\let\beamer@writeslidentry=\beamer@writeslidentry@miniframesoff}
\makeatother



\newcommand{\source}[2]{{\tiny\it Source: \href{#1}{#2}}}

\usepackage[normalem]{ulem}

\usepackage{tikz}
\usetikzlibrary{intersections,arrows,shapes,calc,mindmap,backgrounds,positioning,svg.path}

\tikzset{box/.style={
            draw, 
            fill=blue!20,
            fill opacity=0.8,
            thick,
            inner sep=0pt,
            minimum size=1cm,
            transform shape
        },
        finalbox/.style={
            draw, 
            fill=orange,
            fill opacity=0.8,
            thick,
            inner sep=0pt,
            minimum size=1cm,
            transform shape
        },
        dot/.style={
            draw,
            circle,
            fill=red!20,
            inner sep=0pt,
            minimum size=1cm,
            transform shape
        },
        axis/.style={
            thick,
            gray,
            font=\small},
        every to/.style={
            >=latex,
            dashed,
            thick
        }
    }


\graphicspath{{figs/}{figs/anthropomorphism/}}

\title{Automatic Processing of Emotions}
\subtitle{AINT512}

\date{}
\author{Séverin Lemaignan}
\institute{Centre for Neural Systems and Robotics\\{\bf Plymouth University}}

\begin{document}

\miniframesoff

\licenseframe{github.com/severin-lemaignan/lecture-emotions}

\maketitle


\begin{frame}{This week}

This week, we are looking at the automatic processing of emotions:

    \begin{itemize}
        \item models of emotions
        \item action units
        \item emotion classification
        \item emotion generation
    \end{itemize}

\end{frame}

\miniframeson

\section{Models of emotion}

\begin{frame}{Paul Ekman and basic emotions}

\href{https://en.wikipedia.org/wiki/Paul_Ekman}{Paul Ekman}, a
psychologist, found that when shown facial expressions people across the
world all recognised six \textbf{basic emotions}.

    \only<1>{
    \begin{center}
        \includegraphics[width=0.55\linewidth]{ekman_6_basic_emotions}

        \source{http://emotionresearcher.com/}{emotionresearcher.com}
    \end{center}
}

\pause

\begin{center}
\textbf{Anger, disgust, fear, happiness, sadness and surprise}.
\end{center}

Some other emotion are less universally recognised, e.g.~contempt.

\pause 
In 1990 Ekman extended his list of basic emotion to include other
emotions: \emph{Amusement, contempt, contentment, embarrassment, excitement, guilt,
  pride in achievement, relief, satisfaction, sensory pleasure, and
    shame}


\end{frame}


\begin{frame}{Emotion and machines}

    \only<1-2>{
        {\bf Computer and robots should be able to read emotions}

    \begin{itemize}
        \item This is crucial in understanding the social world around the robot and
            responding appropriately.
        \item This has a large number of applications, e.g.
            \href{https://en.wikipedia.org/wiki/Sentiment_analysis}{sentiment
            analysis} to predict what the
            \href{http://arxiv.org/PS_cache/arxiv/pdf/1010/1010.3003v1.pdf}{stock
            market will do based on emotional content of tweets}.
    \end{itemize}

    \pause

    {\bf Should machines express emotions?}

    \begin{itemize}

        \item Due to the \href{https://en.wikipedia.org/wiki/The_Media_Equation}{\bf
            media equation} we cannot help attributing emotions to
            robots and other technology.
        \item As we expect machines to express emotions, it would be good as a robot
            designer to take control of the unconscious attribution of emotion by
            the user.
        \item + cf Serge's slides
    \end{itemize}

}
    \only<3>{
        {\bf Why would representing emotion be useful?}

    \begin{itemize}

        \item You can model the emotional state of the user and respond
            appropriately. For example, the user is bored, if the robot knows this
            it can do something to entertain the user.
        \item You can use a model to let the robot express emotions. For example,
            this can be used for ``emotion contagion''. If the robot is happy,
            this will influence how the user feels.
    \end{itemize}
}
\end{frame}

\begin{frame}{How to represent emotions?}

\only<1-4>{
    Several approaches (or {\bf models}):
}

\only<1-2>{
ON/OFF approach to basic emotions:

\begin{itemize}

\item Anger : ON/OFF
\item Disgust : ON/OFF
\item Fear : ON/OFF
\item Happiness : ON/OFF
\item Sadness : ON/OFF
\item Surprise : ON/OFF
\end{itemize}

Why is this a poor approach to representing emotion?

\pause

$\Rightarrow$ no representation of gradual emotions; all is just ``on'' or
``off''
}
\only<3-4>{


Gradual representation of emotion would be better:

\begin{itemize}

\item Anger : {[}0; 1{]}
\item Disgust : {[}0; 1{]}
\item Fear : {[}0; 1{]}
\item Happiness : {[}0; 1{]}
\item Sadness : {[}0; 1{]}
\item Surprise : {[}0; 1{]}
\end{itemize}

What is the problem with this representation?

\onslide<4>
$\Rightarrow$ can lead to contradicting emotion representation, for example
Happiness = 1 and Sadness = 1.
}


    \only<5>{
    $\Rightarrow$ continuous representations of emotions in space
}
\end{frame}

{
    \paper{Mollahosseini et al. {\bf Affectnet: A database for facial
    expression, valence, and arousal computing in the wild}, 2017. arXiv
    preprint arXiv:1708.03985.}
\begin{frame}{The circumplex model}
    \begin{center}
        \includegraphics[width=0.65\linewidth]{AffectNet_circumplex}
    \end{center}
\end{frame}
}

\begin{frame}{Representing emotion}

\begin{itemize}

\item \textbf{Emotion circumplex model}
\end{itemize}

(from \href{http://www.ncbi.nlm.nih.gov/pmc/articles/PMC2367156/}{Posner
et al., 2008})

\end{frame}

\begin{frame}{Circumplex model}

\textbf{Emotion circumplex model} developed by Russell

Two-dimensional space (\textbf{arousal} and \textbf{valence} on x and y
dimensions)

\begin{itemize}

\item Emotions are noted on circumference of the circle.
\item Middle of circle (0,0) is \textbf{neutral}
\item The further from the centre, the stronger the emotion.
\end{itemize}

(from \href{http://www.ncbi.nlm.nih.gov/pmc/articles/PMC2367156/}{Posner
et al., 2008})

\end{frame}

\begin{frame}{PAD model}

By Albert Mehrabian and James A. Russell (1974)

Three-dimensional representation of emotion.

\begin{itemize}

\item \textbf{Pleasure, Arousal and Dominance}.
\item Most used model in robotics and Human-Machine Interaction
\end{itemize}

\textbf{Pleasure}: how pleasant is an emotion. Anger and fear are
unpleasant emotions, and score low on the pleasure scale. Joy is a
pleasant emotion.

\textbf{Arousal}: intensity of an emotion. Anger and rage are unpleasant
emotions, but rage has a higher intensity or a higher arousal state.
However boredom, which is also an unpleasant state, has a low arousal
value.

\textbf{Dominance}: controlling and dominant nature of the emotion. For
instance while both fear and anger are unpleasant emotions, anger is a
dominant emotion, while fear is a submissive emotion.

\end{frame}

\begin{frame}{PAD model}

Dominance

Arousal

Pleasure

\textbf{Joy}

\textbf{Tired}

\textbf{Soothed}

\textbf{Surprise}

\textbf{Fear}

\textbf{Anger}

\textbf{Alert}

\textbf{Neutral}

\textbf{Calm}

\textbf{Disgust}

\end{frame}

\section{Action Units}

\newcommand{\au}[1]{
    \includegraphics[width=2cm]{au/au-0#1.jpg}
}


{
    \paper{Ekman, {\bf The Facial Action Coding System}, 1977}
\begin{frame}{Facial Action Coding System}

\end{frame}
}

{
    \paper{\source{https://www.cs.cmu.edu/\%7Eface/facs.htm}{Automated Face Analysis group, CMU}}
\begin{frame}{Action Units}
    \begin{center}

    \scriptsize
        \begin{tabular}{@{}p{0.5cm}p{2.5cm}p{3.5cm}p{2.5cm}@{}}
    \toprule
    \textbf{AU} & \textbf{Description} & \textbf{Facial muscle}                                                                   & \textbf{Example image} \\
    \midrule
    \only<1>{
    \textbf{1}  & Inner Brow Raiser    & \textit{Frontalis, pars medialis}                                                        & \au{01}                       \\
    \textbf{2}  & Outer Brow Raiser    & \textit{Frontalis, pars lateralis}                                                       & \au{02}                       \\
    \textbf{4}  & Brow Lowerer         & \textit{Corrugator supercilii, Depressor supercilii}                                     & \au{04}                       \\
    \textbf{5}  & Upper Lid Raiser     & \textit{Levator palpebrae superioris}                                                    & \au{05}                       \\
    \textbf{6}  & Cheek Raiser         & \textit{Orbicularis oculi, pars orbitalis}                                               & \au{06}                       \\
    \bottomrule
    }
    \only<2>{
    \textbf{7}  & Lid Tightener        & \textit{Orbicularis oculi, pars palpebralis}                                             & \au{07}                       \\
    \textbf{9}  & Nose Wrinkler        & \textit{Levator labii superioris alaquae nasi}                                           & \au{09}                       \\
    \textbf{10} & Upper Lip Raiser     & \textit{Levator labii superioris}                                                        & \au{10}                       \\
    \textbf{11} & Nasolabial Deepener  & \textit{Zygomaticus minor}                                                               & \au{11}                       \\
    \textbf{12} & Lip Corner Puller    & \textit{Zygomaticus major}                                                               & \au{12}                       \\
    \bottomrule
    }
    \only<3>{
    \textbf{13} & Cheek Puffer         & \textit{Levator anguli oris (a.k.a. Caninus)}                                            & \au{13}                       \\
    \textbf{14} & Dimpler              & \textit{Buccinator}                                                                      & \au{14}                       \\
    \textbf{15} & Lip Corner Depressor & \textit{Depressor anguli oris (a.k.a. Triangularis)}                                     & \au{15}                       \\
    \textbf{16} & Lower Lip Depressor  & \textit{Depressor labii inferioris}                                                      & \au{16}                       \\
    \bottomrule
    }
    \only<4>{
    \textbf{17} & Chin Raiser          & \textit{Mentalis}                                                                        & \au{17}                       \\
    \textbf{18} & Lip Puckerer         & \textit{Incisivii labii superioris and Incisivii labii inferioris}                       & \au{18}                       \\
    \textbf{20} & Lip stretcher        & \textit{Risorius w/ platysma}                                                            & \au{20}                       \\
    \textbf{22} & Lip Funneler         & \textit{Orbicularis oris}                                                                & \au{22}                       \\
    \textbf{23} & Lip Tightener        & \textit{Orbicularis oris}                                                                & \au{23}                       \\
    \bottomrule
    }
    \only<5>{
    \textbf{24} & Lip Pressor          & \textit{Orbicularis oris}                                                                & \au{24}                       \\
    \textbf{25} & Lips part            & \textit{Depressor labii inferioris or relaxation of Mentalis, or Orbicularis oris}       & \au{25}                       \\
    \textbf{26} & Jaw Drop             & \textit{Masseter, relaxed Temporalis and internal Pterygoid}                             & \au{26}                       \\
    \textbf{27} & Mouth Stretch        & \textit{Pterygoids, Digastric}                                                           & \au{27}                       \\
    \bottomrule
    }
    \only<6>{
    \textbf{28} & Lip Suck             & \textit{Orbicularis oris}                                                                & \au{28}                       \\
    \textbf{41} & Lid droop            & \textit{Relaxation of Levator palpebrae superioris}                                      & \au{41}                       \\
    \textbf{42} & Slit                 & \textit{Orbicularis oculi}                                                               & \au{42}                       \\
    \textbf{43} & Eyes Closed          & \textit{Relaxation of Levator palpebrae superioris; Orbicularis oculi, pars palpebralis} & \au{43}                       \\
    \textbf{44} & Squint               & \textit{Orbicularis oculi, pars palpebralis}                                             & \au{44}                       \\
    \bottomrule
    }
    \only<7>{
    \textbf{45} & Blink                & \textit{Levator palpebrae superioris; Orbicularis oculi, pars palpebralis} &                               \\
    \textbf{46} & Wink                 & \textit{Levator palpebrae superioris; Orbicularis oculi, pars palpebralis} &                               \\
    \textbf{51} & Head turn left       &                                                                                          & \au{51}                       \\
    \textbf{52} & Head turn right      &                                                                                          & \au{52}                       \\
    \bottomrule
    }
    \only<8>{
    \textbf{53} & Head up              &                                                                                          & \au{53}                       \\
    \textbf{54} & Head down            &                                                                                          & \au{54}                       \\
    %\textbf{55} & Head tilt left       &                                                                                          & \au{55}                       \\
    %\textbf{56} & Head tilt right      &                                                                                          & \au{56}                       \\
    \bottomrule
    }
    \only<9>{
    %\textbf{57} & Head forward         &                                                                                          & \au{57}                       \\
    %\textbf{58} & Head back            &                                                                                          & \au{58}                       \\
    \textbf{61} & Eyes turn left       &                                                                                          & \au{61}                       \\
    \textbf{62} & Eyes turn right      &                                                                                          & \au{62}                       \\
    \textbf{63} & Eyes up              &                                                                                          & \au{63}                       \\
    \textbf{64} & Eyes down            &                                                                                          & \au{64}                       \\ 
    \bottomrule
    }
    \end{tabular}
    \end{center}

\end{frame}
}

\begin{frame}{OpenFace Action Units}
    \begin{center}
        \Large \href{https://github.com/TadasBaltrusaitis/OpenFace}{github.com/TadasBaltrusaitis/OpenFace}
        \vspace{2em}

        \includegraphics[width=\linewidth]{au_openface}

        \scriptsize
        (not to be confused with this other \href{https://github.com/cmusatyalab/openface}{CMU OpenFace})
    \end{center}
\end{frame}

\begin{frame}{OpenFace 18 Action Units}

    \begin{columns}
        \begin{column}{0.5\linewidth}
    \begin{center}

    \scriptsize
        \begin{tabular}{@{}p{0.3cm}p{2.5cm}p{2cm}@{}}
    \toprule
    \textbf{AU} & \textbf{Description} & \textbf{Example image} \\
    \midrule
    \only<1>{
    \textbf{1}  & Inner Brow Raiser    &  \au{01} \\
    \textbf{2}  & Outer Brow Raiser    &  \au{02} \\
    \textbf{4}  & Brow Lowerer         &  \au{04} \\
    \textbf{5}  & Upper Lid Raiser     &  \au{05} \\
    \textbf{6}  & Cheek Raiser         &  \au{06} \\
    \bottomrule
    }
    \only<2>{
    \textbf{15} & Lip Corner Depressor &  \au{15} \\
    \textbf{17} & Chin Raiser          &  \au{17} \\
    \textbf{20} & Lip stretcher        &  \au{20} \\
    \textbf{23} & Lip Tightener        &  \au{23} \\
    \bottomrule
    }
    \end{tabular}
    \end{center}
            
        \end{column}
        \begin{column}{0.5\linewidth}
    \begin{center}

    \scriptsize
        \begin{tabular}{@{}p{0.3cm}p{2cm}p{2cm}@{}}
    \toprule
    \textbf{AU} & \textbf{Description} & \textbf{Example image} \\
    \midrule
    \only<1>{
    \textbf{7}  & Lid Tightener        &  \au{07} \\
    \textbf{9}  & Nose Wrinkler        &  \au{09} \\
    \textbf{10} & Upper Lip Raiser     &  \au{10} \\
    \textbf{12} & Lip Corner Puller    &  \au{12} \\
    \textbf{14} & Dimpler              &  \au{14} \\
    \bottomrule
    }
    \only<2>{
    \textbf{25} & Lips part            &  \au{25} \\
    \textbf{26} & Jaw Drop             &  \au{26} \\
    \textbf{28} & Lip Suck             &  \au{28} \\
    \textbf{45} & Blink                &          \\
    \bottomrule
    }
    \end{tabular}
    \end{center}
         \end{column}
    \end{columns}

\end{frame}

\begin{frame}{Action Units}



\source{https://en.wikipedia.org/wiki/Facial_Action_Coding_System}{Wikipedia}
\end{frame}

\section{Generating emotional responses}

\imageframe{wall-e}

\begin{frame}{How expressive a robot can be?}

    \begin{columns}
        \begin{column}{0.5\linewidth}
            \begin{center}
                \includegraphics[width=0.8\linewidth]{cozmo}
            \end{center}
        \end{column}
        \begin{column}{0.5\linewidth}
            \href{https://www.anki.com/en-gb/cozmo}{Anki's Cozmo}
        \end{column}
    \end{columns}
\end{frame}

\videoframe[0.56]{figs/cozmo.mp4}

\imageframe[color=black]{cozmo-expression-sheet}


\section{Anthropomorphism}

\imageframe[footnote=Source: Christoph Bartneck,scale=0.9]{bartneck}

\imageframe{anthropo}

\begin{frame}{Dynamics of anthropomorphism}
    \begin{center}
        \includegraphics<1>[width=0.8\linewidth]{dynamics-0}
        \includegraphics<2>[width=0.8\linewidth]{dynamics-1}
        \includegraphics<3>[width=0.8\linewidth]{dynamics-2}
        \includegraphics<4>[width=0.8\linewidth]{dynamics-3}
    \end{center}
\end{frame}

\begin{frame}{Cozmo?}

    \begin{columns}
        \begin{column}{0.4\linewidth}
            \begin{center}
                \includegraphics[width=0.8\linewidth]{cozmo}
            \end{center}
        \end{column}
        \begin{column}{0.6\linewidth}
            \begin{center}
                \includegraphics[width=\linewidth]{dynamics-3}
            \end{center}
        \end{column}
    \end{columns}
\end{frame}


\miniframesoff

\begin{frame}{}
    \begin{center}
        \Large
        That's all for today, folks!\\[2em]
        \normalsize
        Questions:\\
        Portland Square B316 or \url{severin.lemaignan@plymouth.ac.uk} \\[1em]

        Slides:\\
        \href{https://github.com/severin-lemaignan/lecture-emotions}{\small
        github.com/severin-lemaignan/lecture-emotions}

    \end{center}
\end{frame}




\end{document}
